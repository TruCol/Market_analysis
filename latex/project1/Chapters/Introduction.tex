\section{Introduction}\label{sec:intro}
Welcome, this document presents our market analysis for the TruCol consultancy. The objective of this document is to provide some basic insight into the order of magnitude of the potential of the TruCol consultancy to generate returns for its potential investors. Based on various pitch templates, \cite{kamps2020}, and private communications, we intend to convey this information through sharing our model and estimate of the following market parameters for the TruCol consultancy:

\begin{itemize}
	\item \textbf{Total addressable market (TAM)}, or total available market, is the total market demand for a product or service, calculated in annual revenue or unit sales if 100\% of the available market is achieved\cite{tam_sam_som}.
	\item \textbf{Serviceable available market (SAM)} is the portion of TAM targeted and served by a company's products or services\cite{tam_sam_som}.
	\item \textbf{Serviceable obtainable market (SOM)}, or share of market, is the percentage of SAM which is realistically reached\cite{tam_sam_som}.
\end{itemize}


\noindent Since we currently have little experience on this topic within our team, we are making our data and assumptions as transparent as possible, both in this document as in our code. This way we hope to improve our model based on your feedback by enabling you to experiment with it yourself. Additionally, because the market analysis consists of a rough estimate, three different estimation methods are used for generating the TAM, SAM and SOM estimates. The redundancy is introduced to establish some frame of reference within the results. % TODO: Improve formulation, referrence frame within the results seems counterintuitive.

The assumptions and data points for the respective models are specified in \cref{sec:assumptions}. Next, the models are described in \cref{sec:model_description} (the Python models themselves are included as appendices in \cref{app:0} to \cref{app:2} respectively). The results of these models are presented in \cref{sec:results}. To shed some light on how sensitive the model is to for example changes in assumptions, a sensitivity analysis is presented for each model in \cref{sec:sensitivity_analysis}. Next the results and sensitivity of the models are discussed in \cref{sec:discussion} and a conclusion is provided in \cref{sec:conclusion}.

We invite you to tinker with the assumptions and models yourself! The data and plots in this report are automatically updated if you run \verb+python -m code.project1.src+. If you experience any difficulties in running the code, simply reach out to us, (click on issues on the github page) and we are happy to get you running the code.
