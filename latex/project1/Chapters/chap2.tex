\section{Model Description}\label{sec:model_description}
Since the market size estimation models are somewhat of an abstract/subjective task, three different approaches are used in an attempt to establish some reference material with respect to accuracy. 

Before the model is presented, it is important to realise that we propose a consultancy service that operates as an optimisation service. This means that if a certain activity, e.g. a logistics company has operational cost of 5 \$million/day, our consultancy service is only able to earn at most the margin of improvement we are able to bring our customer. So suppose the independent usage of the TruCol provides the customer with a 2\% optimisation in their operational costs, yielding them $5.000.000\cdot 0.02=100.000/day$\$. Suppose our expertise is able to enable them to yield a 3\% optimisation by identifying the relevant development/system processes and supporting them in improved test specification. In that assumption our consultancy would bring them an additional 3-2=1\% which would translate roughly to $50.000$\$. That would be the value we bring to the logistics company in this hypothetical scenario.

In reality this example is oversimplified, the 2\% the company could get by themselves would involve some risk pertaining to inaccurate test specification which could lead to loss of the bounty. Our company reduces this risk by providing test-specification security expertise. Furthermore, our interaction with the client may bring the client experience that can be applied in future applications of the TruCol protocol, hence the value to we bring to the client is larger than the amount they gain in terms of optimisation w.r.t. the case where they use the protocol themselves.

\subsection{Top Down}\label{subsec:model_descriptions_top_down}
The Top-Down approach 
\subsection{Bottom Up}\label{subsec:model_descriptions_bottom_up}
\subsection{Value Theory}\label{subsec:model_descriptions_value_theory}