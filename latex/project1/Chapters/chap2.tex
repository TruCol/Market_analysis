\section{Model Description}\label{sec:model_description}
% What to convey (Source: Pitch Perfect):
% What is our Market?
% What is the Market Size?
% What is the Market Trajectory


\subsection{Market}\label{subsec:model_description_market}
To compute the TAM, SAM and SOM, some form of market definition can be used. To this end it can be valuable to specify exactly what the TruCol consultancy does, where it adds value and how it does that. Furthermore, since these three afore mentioned estimates pertain to a potential future, the potential, yet deemed feasible, activities of the TruCol consultancy are included.

The TruCol consultancy provides advice and support to companies on how companies can get the most out of the TruCol protocol. To understand this the following assumptions are shared:

\begin{itemize}
	\item \textbf{asu-0:} Task completions of tasks that are completed using the TruCol protocol are deterministically verifyable.
	\item \textbf{asu-1:} Solutions of tasks that are completed using the TruCol protocol are of sufficient quality.
	\item \textbf{asu-2:} Tasks that are completed using the TruCol protocol can be solved for the lowest costprice that is currently available in this world.
	\item \textbf{asu-3:} No personel needs to be attracted, screened, hired nor fired for tasks that are completed using the TruCol protocol.
	\item \textbf{asu-4:} Companies can benefit from public particular solutions to their task specifications. 
	\item \textbf{asu-5:} By sampling from a bigger talent pool (this world), the average performance of the solutions will be better than what is produced by the in-house talent pool, or, for equal solution performance, a faster rate of development can be obtained on average for an equal or lower price.
\end{itemize}

Based on these assumptions, one can conclude that an economically rational company would try to off-load as much of their required tasks into the TruCol protocol as it would minimise their operational costs. 

We help companies identify the tassks for which they can use the TruCol protocol, and we assist them in writing safe test specifications that are not easily hackable. 

% NP problems
% Neuromorphic
% Space
% Logistics
% Chemical compound development
% Protein folding
% AI requirements specicfication
% AI engines
\subsection{Market Size}\label{subsec:model_description_market_size}
\subsection{Market Trajectory}\label{subsec:model_description_market_trajectory}

Since the market size estimation models are somewhat of an abstract/subjective task, three different approaches are used in an attempt to establish some reference material with respect to accuracy.



Before the model is presented, it is important to realise that we propose a consultancy service that operates as an optimisation service. This means that if a certain activity, e.g. a logistics company has operational cost of 5 \$million/day, our consultancy service is only able to earn at most the margin of improvement we are able to bring our customer. So suppose the independent usage of the TruCol provides the customer with a 2\% optimisation in their operational costs, yielding them $5.000.000\cdot 0.02=100.000/day$\$. Suppose our expertise is able to enable them to yield a 3\% optimisation by identifying the relevant development/system processes and supporting them in improved test specification. In that assumption our consultancy would bring them an additional 3-2=1\% which would translate roughly to $50.000$\$. That would be the value we bring to the logistics company in this hypothetical scenario.

In reality this example is oversimplified, the 2\% the company could get by themselves would involve some risk pertaining to inaccurate test specification which could lead to loss of the bounty. Our company reduces this risk by providing test-specification security expertise. Furthermore, our interaction with the client may bring the client experience that can be applied in future applications of the TruCol protocol, hence the value to we bring to the client is larger than the amount they gain in terms of optimisation w.r.t. the case where they use the protocol themselves.



\subsubsection{Top Down}\label{subsubsec:model_descriptions_top_down}
The Top-Down approach 
\subsubsection{Bottom Up}\label{subsubsec:model_descriptions_bottom_up}
\subsubsection{Value Theory}\label{subsubsec:model_descriptions_value_theory};