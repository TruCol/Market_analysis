

\subsection{TruCol Revenue}
Hence, if each of those companies in the logistics sector could increase their profits on average annually by \textbf{\avgalgooptimisationprofitpercentage}\% using algorithmic optimisation, and if they would use the TruCol protocol to do that, and if they would be willing to invest \textbf{\TruColcutonprofitpercentage}\% of that profit in our support and assistance in getting the most out of the TruCol protocol, we would currently estimate that this would yield roughly an income of $\textcolor{blue}{74.2}\cdot \textbf{\avgalgooptimisationprofit}\cdot \textbf{\TruColcutonprofit}=\$\textcolor{blue}{0.74} million$

\subsection{Additional addressable markets}\label{subsubsec:additional_markets}
Since the TruCol company is market agnostic, we also seek to assist in algorithmic optimisation outside the logistics market. Several markets are worth mentioning in particular as we expect them to either heavily rely on algorithmic optimisations, or because they are particularly suited for the TruCol protocol.
\begin{itemize}
	\item \textbf{(Automated) trading} In the highly competitive market of (automated) trading, algorithmic optimisations are key to making successful trades.
	\item \textbf{Space Sector} The space engineering sector already has a relatively high test driven development\cite{todo}, this lowers the adoption costs of the TruCol protocol relative to most industries. Furthermore, space applications are heavily mass constrained, which generally makes them highly energy constrained as well. These energy constraints emphasise the importance of algorithmic optimisations, for example in telecommunications satellites and swarm robots.
	\item \textbf{Innovative Materials Research} The domain of material science has been adopting algorithmic search strategies to find new materials  \cite{allahyari2020coevolutionary}.
	\item \textbf{Pharmaceutical Industry} Another example of a large market that has been shifting to adopt algorithmic search strategies to find new medicines.
\end{itemize}
Each of these are multi-billion dollar markets which can contribute to the TAM of the TruCol company.
% NP problems
% Neuromorphic
% Space
% Logistics
% Chemical compound development
% Protein folding
\subsubsection{Emerging markets}\label{subsubsec:emerging_markets}
Beyond those listed markets, the following emerging markets could be great opportunities for the TruCol company to latch in and grow along in.
\begin{itemize}
	\item \textbf{Neuromorphic Computing} This field is developing new complexity theory to adapt to the unconventional computation methods. This is an interesting opportunity to explore the versatility of the TruCol protocol.
	\item \textbf{Quantum Computing} This is another upcoming field with many new algorithmic implementations. The newness of the field may suggest that the amount of optimisation and exploration to be done is relatively high, possibly indicating a relatively large potential for the TruCol protocol. However, currently our team does not yet contain experience in this type of algorithmic developments.
	\item \textbf{Artificial Intelligence} With the introduction of GPT-3 and GitHub Copilot, the world has seen examples of AI engines that are able to generate code for some basic tasks. The TruCol protocol could catalyse the usage of such AI engines that are able to write code based on requirement specification. We expect that users of the TruCol protocol will develop a tactical advantage on requirement specifications for AI engines.
\end{itemize}

% AI requirements specicfication
% AI engines

%\subsection{Market Trajectory}\label{subsec:model_description_market_trajectory}
